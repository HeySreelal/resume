\documentclass[11pt,a4paper,sans]{moderncv}
\moderncvstyle{banking}
\moderncvcolor{black}
\nopagenumbers{}
\usepackage[utf8]{inputenc}
\usepackage{ragged2e}
\usepackage[scale=0.915]{geometry}
\usepackage{import}
\usepackage{multicol}
\usepackage{import}
\usepackage{enumitem}
\usepackage[utf8]{inputenc}
\usepackage{amssymb}
\name{Sreelal}{TS}
\newcommand*{\customcventry}[7][.13em]{
\begin{tabular}{@{}l}
{\bfseries #4} \
{\itshape #3}
\end{tabular}
\hfill
\begin{tabular}{l@{}}
{\bfseries #5} \
{\itshape #2}
\end{tabular}
\ifx&#7&%
\else{\
\begin{minipage}{\maincolumnwidth}%
\small#7%
\end{minipage}}\fi%
\par\addvspace{#1}}
\begin{document}
\makecvtitle
\vspace*{-16mm}
\begin{center}\textbf{ Flutter Developer}\end{center}
\begin{center}
    \begin{tabular}{ c c c }
        \faMobile\enspace +91 9745521246                                                              & \enspace  \href{mailto:heysreelal@gmail.com}{heysreelal@gmail.com}                             & \faHome\enspace Kerala, India \\
        \faLinkedin\enspace \color{blue} \href{https://www.linkedin.com/in/heysreelal}{in/heysreelal} &
        \faGithub\enspace \color{blue} \href{https://github.com/heysreelal}{github.com/heysreelal}    & \enspace {$\mathbb{X}$}\enspace \color{blue} \href{https://x.com/heysreelal}{x.com/heysreelal}
    \end{tabular}
\end{center}

\section{Profile}
 {Passionate Flutter Developer with a knack for crafting beautiful and functional mobile applications. Adept at building cross-platform experiences using the latest Dart and Flutter features. Proven ability to deliver high-performance apps that exceed user expectations. Strong problem-solving skills and a collaborative spirit ensure successful project delivery.}

\section{Areas of Expertise}
 {Dart Programming - Flutter Development - Firebase Integration - NodeJS Development - UI/UX Design - RESTful API Development - Agile Development - Database Management - Continuous Integration/Continuous Deployment (CI/CD)}

\section{Professional Experience}
\customcventry{07/2020 ‐ 01/2024}{{\color{blue}\href{https://pokitqr.com/}{Pokit QR}}}{Flutter Developer,}{Kozhikode, Kerala}{}{
    {\begin{itemize}[leftmargin=0.6cm, label={\textbullet}]
                \item Led the development and launch of Pokit QR, a professional networking app, from conceptualization to deployment.
                \item Implemented innovative features including QR-based connections, nearby friend discovery, and comprehensive privacy controls, enhancing user experience and engagement
                \item Managed the seamless integration of various authentication channels such as Google, Phone Number, and email, ensuring users enjoy a secure and hassle-free login experience.
                \item Implemented Continuous Integration and Continuous Delivery (CI/CD) pipelines for Pokit QR, ensuring automated testing, seamless deployment, and efficient release cycles to maintain high-quality standards and timely updates on both the Play Store.
            \end{itemize}}}

\customcventry{01/2021 ‐ 01/2024}{{\color{blue}\href{https://pokitqr.com/}{Pokit QR}}}{Director,}{Kozhikode, Kerala}{}{
    {\begin{itemize}[leftmargin=0.6cm, label={\textbullet}]
                \item Directed the strategic vision and development roadmap for Pokit QR, overseeing all aspects of the app's design, functionality, and user experience
                \item Led a cross-functional team of developers, designers, and marketers to ensure timely delivery of features and enhancements, fostering collaboration and innovation
            \end{itemize}}}

\customcventry{02/2022 ‐ 03/2024}{{\color{blue}\href{https://www.linkedin.com/company/ycorp}{Ycorp}}}{Flutter Developer,}{Kozhikode, Kerala}{}{
    {\begin{itemize}[leftmargin=0.6cm, label={\textbullet}]
                \item Developed and launched the VR Mandir app using Flutter, serving as a pooja booking platform for devotees, with a Django API backend, ensuring a seamless user experience.
                \item Spearheaded the development of the {\color{blue}{\href{https://ywork.in/}{Ywork apps}}}, built with Flutter and featuring a RESTful API made with Golang, streamlining the shopping experience for users in shopping malls
                \item Collaborated with multiple teams at Ycorp startup to deliver high-quality apps, leveraging Flutter for frontend development, animations, and state management and integrating with Django and Golang backends.
            \end{itemize}}}

\customcventry{02/2022 ‐ 01/2023}{{\color{blue}\href{https://cyberdome.kerala.gov.in/}{CyberDome, Kerala Police}}}{Voice Developer,}{Kozhikode, Kerala}{}{
    {\begin{itemize}[leftmargin=0.6cm, label={\textbullet}]
                \item Collaborated with CyberDome, Kerala Police to develop the Kerala Police Assistant, a voice-enabled chat assistant, providing accessibility to Kerala Police services through voice commands.
                \item Integrated the chatbot with Google Assistant using the Actions on Google platform, enabling seamless interaction and access to police services via voice commands.
                \item Utilized NodeJS as a custom backend to efficiently handle user requests, ensuring optimal performance and responsiveness of the chat assistant.
            \end{itemize}}}

\section{Education}
\customcventry{2021-2024}{\color{blue}\href{https://kannuruniversity.ac.in/}{Kannur University}}{BSc Computer Science}{Mananthavady, Kerala}{}{}
{Appeared for BSc Computer Science at Mary Matha Arts \& Science College, Mananthavady affiliated to Kannur University. Relevant Courses: Software Engineering, Programming in C, C++, JAVA and Python, Database Management Systems, Data Structures, Web Technologies, Operating Systems, Computer Networks.}

\clearpage

\section{Skills, Languages, and Tools}
 {\begin{itemize}[label=\textbullet]
      \item {\textbf{Flutter}}: Proficient in Flutter for cross-platform development, incorporating animations to create high-quality apps.
      \item {\textbf{Programming Languages}}: Experienced with Dart, NodeJS, and Golang. Built APIs with both NodeJS and Dart.
      \item {\textbf{Familiar with}}: Familiarity with Figma, Git version control, CI/CD pipelines, Flutter, React, and Firebase.
      \item {\textbf{Problem Solving}}: Strong critical thinking and problem-solving skills.
      \item {\textbf{Soft Skills:} Presentation, Time Management, Quick Learner, Planning, Organized, Creative Problem-Solving, Teamwork, Active Listening, Adaptability, Analytical Thinking}
  \end{itemize}}

\section{Projects}
\begin{itemize}[label=\textbullet]
    \item \textbf{Televerse} - Advanced Telegram Bot API wrapper written in Dart. Televerse can be used to build high-quality Telegram Bots with Dart. Televerse provides a feature-rich API for building Telegram Bots, including support for inline queries, custom keyboards, sessions and more. \\
          Project Link: \textcolor{blue}{\href{https://pub.dev/packages/televerse}{pub.dev/packages/televerse}} | Source Code: \textcolor{blue}{\href{https://github.com/heysreelal/televerse}{github.com/heysreelal/televerse}}

    \item \textbf{Pokit QR} - A professional networking app that allows users to connect with professionals using QR codes. The app let's users create a digital business card and share it with others using a QR code. The app primarily focuses on skill discovery and professional networking. \\
          Project Link: \textcolor{blue}{\href{https://pokitqr.com/}{pokitqr.com}} | \textcolor{blue}{\href{https://play.google.com/store/apps/details?id=com.pokitqr.pokitqr}{Google Play}}

    \item \textbf{Edakkal Caves Ticket Booking System} - An intuitive ticket booking system for Edakkal Caves, Wayanad, Kerala. The system is built with ReactJS and Firebase. The system also features a mobile app for administrators to manage bookings. The system was built for the Wayanad District Tourism Promotion Council (DTPC Wayanad) under the Kerala Tourism Department (Government of Kerala). \\
          Project Link: \textcolor{blue}{\href{https://edakkalcaves.web.app/}{edakkalcaves.web.app}}

    \item \textbf{Wordle Telegram Bot} - A Telegram Bot that allows users to play the popular Wordle game within Telegram. The bot was initially built with NodeJS and migrated to Dart using Televerse. The bot also features a leaderboard and a daily puzzle. Bot is currently offline and can be brought online on request. The bot had over 3000 active players at its peak. \\
          Project Link: \textcolor{blue}{\href{https://t.me/xwordlebot}{t.me/xwordlebot}} | Source Code: \textcolor{blue}{\href{https://github.com/HeySreelal/xwordle}{github.com/HeySreelal/xwordle}}
\end{itemize}

\section{Activities}

\begin{itemize}[label=\textbullet]
    \item \textbf{Platinum Expert on Google Product Experts} -
          Acknowledged by Google as a Platinum Product Expert within the Google Product Experts program, offering support and guidance to users of Google products like Chrome and Google Assistant. Participated in and attended Product Expert Summits held in Singapore in 2022 and in London, UK in 2023.
    \item \textbf{Open Source Contributor} - Actively contribute to various open-source projects, Televerse, Firebase Admin SDK for Dart, TeleDart, Voicy Bot, and more.
    \item \textbf{Community Leader} - I am the organizer for the Flutter Kozhikode community, organizing meetups, workshops, and hackathons to help developers learn and grow. Flutter Kozhikode is directly supported by Google as part of the Flutter Meetup Network program.
    \item \textbf{Mentor} - Mentored junior developers and students, providing guidance and support to help them grow and succeed in their careers. I also enjoy answering questions on StackOverflow and other developer groups.
    \item \textbf{Speaker} - Presented talks at DevFest 2019 organized by GDG Kozhikode and GDG Coimbatore, sharing expertise and insights on Flutter, Dart, and Actions on Google development. Additionally, delivered presentations at Flutter Kozhikode community meetups.
\end{itemize}


\end{document}